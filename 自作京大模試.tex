\documentclass{article}
\usepackage{amsmath,amssymb}%
\usepackage[dvipdfmx]{graphicx}%
\setlength{\fboxrule}{1.5pt}
\begin{document}
\pagestyle{empty} 
\begin{center}
{\LARGE 令\ 和\ 3\ 年\ 度\ O\ B\ K\ 式\ 模\ 試\ 問\ 題 }\vspace{0.5in}\\*
{\huge 数\ \ \ \ \ 学\ \ \ \ (理系)}\vspace{0.5in}\\*
200点満点\vspace{0.1in}\\*
 $\ll$ 配点は, 問題用紙に記載のとおり. $\gg$ \vspace{0.5in}\\*
\end{center}
\begin{flushleft}
(注\ \ \ 意)\\
\end{flushleft}
1. 問題冊子及び解答用紙は投稿されるまで開かないこと.\vspace{0.1in}\\*
2. 解答冊子は表紙のほかに, 各自で用意した各種ページ含めて16ページある\vspace{0.1in}\\
\ \ \ \ ことを想定する.\vspace{0.1in}\\*
3. 問題は全部で6題ある(1ページから2ページ).\vspace{0.1in}\\*
4. 難易度は京大の易からやや易レベル中心のつもりである.\vspace{0.1in}\\*
5. 解答は解答冊子の指定された解答用ページに書くこと. ただし, 続き方をはっ\vspace{0.1in}\\
\ \ \ \ きり示して見開きに隣接する計算用ページに解答の続きを書いてもよい. その\vspace{0.1in}\\
\ \ \ \ 場合は, 解答用ページに「計算用ページに続く」旨を記すこと. このときに限っ\vspace{0.1in}\\
\ \ \ \ て, 計算用ページに書かれているものを解答の一部として採点する. また, 余\vspace{0.1in}\\
\ \ \ \ 白ページに書かれたものは採点の対象としない.\vspace{0.1in}\\*
6. 解答のための下書き, 計算などは, 計算用ページまたは余白ページに書いて,\vspace{0.1in}\\
\ \ \ \ 残しておいてもよい.\vspace{0.1in}\\*
7. 解答に関係のないことを書いた答案は無効にすることがある.\vspace{0.1in}\\*
8. 解答冊子は, どのページも切り離してはならない.\vspace{0.1in}\\*
9. 解答は後日投稿するので, 答案を送る必要はない. \vspace{0.5in}\\*
\begin{flushright}
{\scriptsize $\diamond$ M4(738---25)}
\end{flushright}

\newpage
\begin{tabbing}
\hspace{0.91\textwidth} \= \hspace{0.8\textwidth} \= \kill
\textgt{\fbox{\Large 1}}\> (35点)\>\\
\end{tabbing}
$k$を正の整数, $p,\ q$を素数とする. このとき$(pq)^p-q^q+q=kp^q$を満たす$p,\ q$は\vspace{0.1in}\\*
存在しないことを示せ.\vspace{1.0in}\\*

\begin{tabbing}
\hspace{0.91\textwidth} \= \hspace{0.8\textwidth} \= \kill
\textsf{\fbox{\Large 2}}\> (35点)\>\\
\end{tabbing}
三角形OA$_1$B$_1$の重心をG$_1$とする. 半直線OA$_1,$ OB$_1$上に任意の$n$個の点をと\vspace{0.1in}\\* 
り, それぞれの点をA$_n,$\ B$_n$とし, 三角形OA$_n$B$_n$の重心をG$_n$とする.このとき \vspace{0.1in}\\*
G$_2,$\ G$_3,\ ...,\ $G$_n$が半直線OG$_1$上に存在するならば, 三角形OA$_1$B$_1$と三角形\vspace{0.1in}
OA$_k$B$_k\ (2\leqq k\leqq n)$は相似な三角形であることを示せ.\vspace{1.0in}\\

\begin{tabbing}
\hspace{0.91\textwidth} \= \hspace{0.8\textwidth} \= \kill
\textsf{\fbox{\Large 3}}\> (30点)\>\\
\end{tabbing}
$\alpha,\ \beta,\ \gamma$は$0\textless\alpha,\ 0\textless\beta,\ 0\textless\gamma,\ \alpha+\beta+\gamma=\pi$を満たす実数とする. このとき,\vspace{0.1in}\\* cos$\alpha$cos$\beta$cos$\gamma$の最大値を求めよ. \vspace{1.6in}\\

\hspace{2.14in}{\Large ---$1$---}\hspace{1.3in}{\scriptsize $\diamond$ M4(738---26)}
\newpage
\begin{tabbing}
\hspace{0.91\textwidth} \= \hspace{0.8\textwidth} \= \kill
\textsf{\fbox{\Large 4}}\> (35点)\>\\
\end{tabbing}
$a,b$は$0\textless a,0\textless b,\ 1\textless a+b\leqq \sqrt{2}$を満たす実数とする. このとき, \vspace{0.1in}\\*
$-1\textless \frac{b(2a^2-1)}{2-b}+\frac{a(2b^2-1)}{2-a}\leqq 0$であることを示せ.\vspace{1.0in}\\*

\begin{tabbing}
\hspace{0.91\textwidth} \= \hspace{0.8\textwidth} \= \kill
\textsf{\fbox{\Large 5}}\> (30点)\>\\
\end{tabbing}
正$n$角系Dの外接円Pの半径をR, 内接円Qの半径をrとし, DとPの接点を時\vspace{0.1in}\\
計回りにそれぞれ$A_1,A_2, ...A_n$. DとQの接点を時計回りにそれぞれ\vspace{0.1in}\\$B_1, B_2,...B_n$とする. ここで弧$A_kA_{k+1}(1\leqq k\leqq n)$とDで囲まれた部分の面積\vspace{0.1in}\\を$S_n$. 弧$B_kB_{k+1}$とDで囲まれた部分の面積$T_n$としたとき$\displaystyle \lim_{n\to \infty}n(S_n-T_n)$を\vspace{0.1in}\\求めよ.\vspace{1.1in}\\*

\begin{tabbing}
\hspace{0.91\textwidth} \= \hspace{0.8\textwidth} \= \kill
\textsf{\fbox{\Large 6}}\> (35点)\>\\
\end{tabbing}
$0\textless \theta\textless \frac{\pi}{2}$として, 次のように定義される数列\{$a_n$\}\ $(n=1,2,3,...)$を\vspace{0.1in}\\
$a_1=cos{\frac{\pi}{4}},\ a_{n+1}=2a_n(\frac{2cos\frac{\theta}{2^n}tan\frac{\theta}{2^n}}{1+cos\frac{\theta}{2^n}}) \cdots \ (\ast)$\vspace{0.1in}\\*
としたとき, $\displaystyle \lim_{n \to \infty}\frac{a_n}{sin\frac{\theta}{2^{n+1}}}$を求めよ.\vspace{0.2in}\\
\begin{center}
{\Large 問題は, このページで終わりである.}\vspace{0.2in}\\
\end{center}
\hspace{2.14in}{\Large ---$2$---}\hspace{1.55in}{\scriptsize $\diamond$ M4(738---27)}


\end{document}